%\iffalse meta-comment
%<*driver>
           \ProvidesFile{MyriadPro.dtx}        [2010/11/01 v0.1alpha MyriadPro support for LaTeX]
%</driver>
%<style>   \NeedsTeXFormat{LaTeX2e}
%<style>   \ProvidesPackage{MyriadPro}         [2010/11/01 v0.1alpha MyriadPro support for LaTeX]
%<mtcfg>   \ProvidesFile{mt-MyriadPro.cfg}     [2010/11/01 v0.1alpha microtype config file for MyriadPro]
%<fontdef> \ProvidesFile{MyriadPro-FontDef.sty}[2010/11/01 v0.1alpha font definition macros]
%<Uextra>  \ProvidesFile{UMyriadPro-Extra.fd}  [2010/11/01 v0.1alpha font definitions for U/MyriadPro-Extra.]
%<LGR&lf>  \ProvidesFile{LGRMyriadPro-LF.fd}   [2010/11/01 v0.1alpha font definitions for LGR/MyriadPro-LF.]
%<LGR&osf> \ProvidesFile{LGRMyriadPro-OsF.fd}  [2010/11/01 v0.1alpha font definitions for LGR/MyriadPro-OsF.]
%<LGR&tlf> \ProvidesFile{LGRMyriadPro-TLF.fd}  [2010/11/01 v0.1alpha font definitions for LGR/MyriadPro-TLF.]
%<LGR&tosf>\ProvidesFile{LGRMyriadPro-TOsF.fd} [2010/11/01 v0.1alpha font definitions for LGR/MyriadPro-TOsF.]
%<LGI&lf>  \ProvidesFile{LGIMyriadPro-LF.fd}   [2010/11/01 v0.1alpha font definitions for LGI/MyriadPro-LF.]
%<LGI&osf> \ProvidesFile{LGIMyriadPro-OsF.fd}  [2010/11/01 v0.1alpha font definitions for LGI/MyriadPro-OsF.]
%<LGI&tlf> \ProvidesFile{LGIMyriadPro-TLF.fd}  [2010/11/01 v0.1alpha font definitions for LGI/MyriadPro-TLF.]
%<LGI&tosf>\ProvidesFile{LGIMyriadPro-TOsF.fd} [2010/11/01 v0.1alpha font definitions for LGI/MyriadPro-TOsF.]
%<OML&lf>  \ProvidesFile{OMLMyriadPro-LF.fd}   [2010/11/01 v0.1alpha font definitions for OML/MyriadPro-LF.]
%<OML&osf> \ProvidesFile{OMLMyriadPro-OsF.fd}  [2010/11/01 v0.1alpha font definitions for OML/MyriadPro-OsF.]
%<OML&tlf> \ProvidesFile{OMLMyriadPro-TLF.fd}  [2010/11/01 v0.1alpha font definitions for OML/MyriadPro-TLF.]
%<OML&tosf>\ProvidesFile{OMLMyriadPro-TOsF.fd} [2010/11/01 v0.1alpha font definitions for OML/MyriadPro-TOsF.]
%<OT1&lf>  \ProvidesFile{OT1MyriadPro-LF.fd}   [2010/11/01 v0.1alpha font definitions for OT1/MyriadPro-LF.]
%<OT1&osf> \ProvidesFile{OT1MyriadPro-OsF.fd}  [2010/11/01 v0.1alpha font definitions for OT1/MyriadPro-OsF.]
%<OT1&tlf> \ProvidesFile{OT1MyriadPro-TLF.fd}  [2010/11/01 v0.1alpha font definitions for OT1/MyriadPro-TLF.]
%<OT1&tosf>\ProvidesFile{OT1MyriadPro-TOsF.fd} [2010/11/01 v0.1alpha font definitions for OT1/MyriadPro-TOsF.]
%<T1&lf>   \ProvidesFile{T1MyriadPro-LF.fd}    [2010/11/01 v0.1alpha font definitions for T1/MyriadPro-LF.]
%<T1&osf>  \ProvidesFile{T1MyriadPro-OsF.fd}   [2010/11/01 v0.1alpha font definitions for T1/MyriadPro-OsF.]
%<T1&tlf>  \ProvidesFile{T1MyriadPro-TLF.fd}   [2010/11/01 v0.1alpha font definitions for T1/MyriadPro-TLF.]
%<T1&tosf> \ProvidesFile{T1MyriadPro-TOsF.fd}  [2010/11/01 v0.1alpha font definitions for T1/MyriadPro-TOsF.]
%<LY1&lf>  \ProvidesFile{LY1MyriadPro-LF.fd}   [2010/11/01 v0.1alpha font definitions for LY1/MyriadPro-LF.]
%<LY1&osf> \ProvidesFile{LY1MyriadPro-OsF.fd}  [2010/11/01 v0.1alpha font definitions for LY1/MyriadPro-OsF.]
%<LY1&tlf> \ProvidesFile{LY1MyriadPro-TLF.fd}  [2010/11/01 v0.1alpha font definitions for LY1/MyriadPro-TLF.]
%<LY1&tosf>\ProvidesFile{LY1MyriadPro-TOsF.fd} [2010/11/01 v0.1alpha font definitions for LY1/MyriadPro-TOsF.]
%<T5&lf>   \ProvidesFile{T5MyriadPro-LF.fd}    [2010/11/01 v0.1alpha font definitions for T5/MyriadPro-LF.]
%<T5&osf>  \ProvidesFile{T5MyriadPro-OsF.fd}   [2010/11/01 v0.1alpha font definitions for T5/MyriadPro-OsF.]
%<T5&tlf>  \ProvidesFile{T5MyriadPro-TLF.fd}   [2010/11/01 v0.1alpha font definitions for T5/MyriadPro-TLF.]
%<T5&tosf> \ProvidesFile{T5MyriadPro-TOsF.fd}  [2010/11/01 v0.1alpha font definitions for T5/MyriadPro-TOsF.]
%<T2A&lf>  \ProvidesFile{T2AMyriadPro-LF.fd}   [2010/11/01 v0.1alpha font definitions for T2A/MyriadPro-LF.]
%<T2A&osf> \ProvidesFile{T2AMyriadPro-OsF.fd}  [2010/11/01 v0.1alpha font definitions for T2A/MyriadPro-OsF.]
%<T2A&tlf> \ProvidesFile{T2AMyriadPro-TLF.fd}  [2010/11/01 v0.1alpha font definitions for T2A/MyriadPro-TLF.]
%<T2A&tosf>\ProvidesFile{T2AMyriadPro-TOsF.fd} [2010/11/01 v0.1alpha font definitions for T2A/MyriadPro-TOsF.]
%<T2B&lf>  \ProvidesFile{T2BMyriadPro-LF.fd}   [2010/11/01 v0.1alpha font definitions for T2B/MyriadPro-LF.]
%<T2B&osf> \ProvidesFile{T2BMyriadPro-OsF.fd}  [2010/11/01 v0.1alpha font definitions for T2B/MyriadPro-OsF.]
%<T2B&tlf> \ProvidesFile{T2BMyriadPro-TLF.fd}  [2010/11/01 v0.1alpha font definitions for T2B/MyriadPro-TLF.]
%<T2B&tosf>\ProvidesFile{T2BMyriadPro-TOsF.fd} [2010/11/01 v0.1alpha font definitions for T2B/MyriadPro-TOsF.]
%<T2C&lf>  \ProvidesFile{T2CMyriadPro-LF.fd}   [2010/11/01 v0.1alpha font definitions for T2C/MyriadPro-LF.]
%<T2C&osf> \ProvidesFile{T2CMyriadPro-OsF.fd}  [2010/11/01 v0.1alpha font definitions for T2C/MyriadPro-OsF.]
%<T2C&tlf> \ProvidesFile{T2CMyriadPro-TLF.fd}  [2010/11/01 v0.1alpha font definitions for T2C/MyriadPro-TLF.]
%<T2C&tosf>\ProvidesFile{T2CMyriadPro-TOsF.fd} [2010/11/01 v0.1alpha font definitions for T2C/MyriadPro-TOsF.]
%<TS1&lf>  \ProvidesFile{TS1MyriadPro-LF.fd}   [2010/11/01 v0.1alpha font definitions for TS1/MyriadPro-LF.]
%<TS1&osf> \ProvidesFile{TS1MyriadPro-OsF.fd}  [2010/11/01 v0.1alpha font definitions for TS1/MyriadPro-OsF.]
%<TS1&tlf> \ProvidesFile{TS1MyriadPro-TLF.fd}  [2010/11/01 v0.1alpha font definitions for TS1/MyriadPro-TLF.]
%<TS1&tosf>\ProvidesFile{TS1MyriadPro-TOsF.fd} [2010/11/01 v0.1alpha font definitions for TS1/MyriadPro-TOsF.]
%<X2&lf>   \ProvidesFile{X2MyriadPro-LF.fd}    [2010/11/01 v0.1alpha font definitions for X2/MyriadPro-LF.]
%<X2&osf>  \ProvidesFile{X2MyriadPro-OsF.fd}   [2010/11/01 v0.1alpha font definitions for X2/MyriadPro-OsF.]
%<X2&tlf>  \ProvidesFile{X2MyriadPro-TLF.fd}   [2010/11/01 v0.1alpha font definitions for X2/MyriadPro-TLF.]
%<X2&tosf> \ProvidesFile{X2MyriadPro-TOsF.fd}  [2010/11/01 v0.1alpha font definitions for X2/MyriadPro-TOsF.]
%<OT2&lf>  \ProvidesFile{OT2MyriadPro-LF.fd}   [2010/11/01 v0.1alpha font definitions for OT2/MyriadPro-LF.]
%<OT2&osf> \ProvidesFile{OT2MyriadPro-OsF.fd}  [2010/11/01 v0.1alpha font definitions for OT2/MyriadPro-OsF.]
%<OT2&tlf> \ProvidesFile{OT2MyriadPro-TLF.fd}  [2010/11/01 v0.1alpha font definitions for OT2/MyriadPro-TLF.]
%<OT2&tosf>\ProvidesFile{OT2MyriadPro-TOsF.fd} [2010/11/01 v0.1alpha font definitions for OT2/MyriadPro-TOsF.]
%<*driver>
\documentclass{ltxdoc}
\usepackage[T1]{fontenc}
\usepackage[expansion=false]{microtype}
\usepackage[myriadint]{MyriadPro}
\renewcommand{\familydefault}{\sfdefault}
\usepackage[toc,enum,lineno]{tabfigures}
\usepackage[greek,british]{babel}
\usepackage{array,booktabs,pifont,paralist,calc}
\usepackage[unicode,bookmarks]{hyperref}
\hypersetup{
  bookmarksnumbered,
  colorlinks=false,
  pdfborder={0 0 0},
  pdftitle={MyriadPro Support for LaTeX},
  pdfauthor={Achim Blumensath, Andreas B\"uhmann, Michael Zedler,
    Sebastian Schubert},
  pdfkeywords={LaTeX, MyriadPro, MdSymbol}
}

\IfFileExists{lmodern.sty}{%
  \renewcommand{\ttdefault}{lmtt}%
  \DeclareMathAlphabet\mathsf    {T1}{lmss}{m}{n}%
}{}

\newcommand*\pkg[1]{\mbox{\textsf{#1}}}
\newcommand*\file[1]{\textsf{#1}}
\newcommand*\acro[1]{\MakeLowercase{#1}}
\newcommand\pdfTeX{pdf\TeX}
\pdfstringdefDisableCommands{%
  \renewcommand*\acro[1]{#1}%
}

\newcommand*\?{\kern0.08em}

\useshorthands{:}\defineshorthand{:}{\?\string:}
\useshorthands{;}\defineshorthand{;}{\?\string;}

\makeatletter
\newcommand*\option{\@ifstar\option@default\option@}
\newcommand*\option@default[1]{\option@{#1}*}
\newcommand*\option@[1]{\textsf{#1}}
\makeatother

\setlength{\leftmargini}{1em}
\setlength{\parindent}{1em}
\newcommand\tabindent{\noindent\hspace{\leftmargini}}
\newlength\optionswidth
\providecommand\newblock{}
\newenvironment{options}[1][{\makebox[\optionswidth]{}}]{%
  \settowidth{\optionswidth}{#1}%
  \global\optionswidth=\optionswidth
  \list{}{%
    \setlength{\labelwidth}{\optionswidth}%
    \setlength{\labelsep}{2\tabcolsep}%
    \setlength{\leftmargin}{\leftmargini+\labelwidth+\labelsep}%
    \setlength{\rightmargin}{0pt}%
    \setlength{\topsep}{\medskipamount}%
    \setlength{\parsep}{0pt}%
    \setlength{\itemsep}{0pt}%
    \renewcommand*\makelabel[1]{%
      \parbox[t]{\labelwidth}{\raggedright\hspace{0pt}##1}}%
    \renewcommand\newblock{\medskip}%
    \raggedright
  }%
}{%
  \endlist
}
\newenvironment{code}{\begin{quote}\small}{\end{quote}}
%\renewcommand\theCodelineNo{%
%  \normalfont\scriptsize\figureversion{tabular}\arabic{CodelineNo}}
\settowidth\MacroIndent{\normalfont\scriptsize\figureversion{tabular}0000\ }
\renewcommand\Module[1]{%
  \mathcode`\|="2353 \mathcode`\&="2026 $\langle\mathsf{#1}\rangle$}

\setcounter{secnumdepth}{2}
\setcounter{tocdepth}{2}

\begin{document}
\DocInput{MyriadPro.dtx}
\end{document}
%</driver>
%\fi
%
% \GetFileInfo{MyriadPro.dtx}
%
% \title{MyriadPro Support for \LaTeX}
% \author{Achim Blumensath \and Andreas B\"uhmann \and Michael Zedler
% \and Sebastian Schubert}
% \date{\fileversion\ -- \filedate}
% \maketitle
% \tableofcontents
%
% \section{Overview}
%
% The \pkg{MyriadPro} package provides support for the MyriadPro
% font family from Adobe.
% You can use these fonts in a \LaTeX\ document by adding the command
% \begin{quote}
%   |\usepackage{MyriadPro}|
% \end{quote}
% to the preamble.
% This will change both the text font and the math font to MyriadPro.
% If you prefer another math font (such as \pkg{eulervm}) use the option
% \option{onlytext} as explained in Section~\ref{sec:Options}.
%
% \section{Interference with other packages}
% \label{sec:Interferences}
%
% The \pkg{MyriadPro} package automatically loads the following packages:
% \pkg{textcomp}, \pkg{amsmath}, \pkg{fontaxes} and \pkg{MdSymbol}.
% If you want to pass options to these packages you can
% either put the corresponding |\usepackage| command before the
% |\usepackage{MyriadPro}| or you can include the options in the
% |\documentclass| command. The \pkg{MyriadPro} package is \emph{not}
% compatible with \pkg{amssymb} and \pkg{amsfonts}. Please see also the
% corresponding section in the \pkg{MdSymbol} documentation.
%
% The MyriadPro package includes support files for the \pkg{microtype} package
% (version~1.8 or higher), consult the package's documentation for further details.
%
% There is also a slight incompatibility with the \pkg{dcolumn} package
% which expects all figures to have the same width. If you want to use this
% package you either have to specify the \option{mathtabular} option (this is
% the brute force solution, not recommended), or you can use the
% |\figureversion{tabular}| command to switch to tabular figures in front of
% every table (much better, but also more work). In addition, \pkg{dcolumn}
% sets figures in math mode, hence the choice of math figures
% (see Section~\ref{sec:Options}) determines if text or lining figures are used.
%
% \section{Options}\label{sec:Options}
%
% \subsubsection*{Font selection}
%
% The following options specify which version of the fonts you want to use.
% The default settings are marked with an asterisk*.
%
% \begin{options}[\option{proportional}, \option{prop}]
% \item[\option*{smallfamily}] use only regular and bold face
% \item[\option{medfamily}]    use semibold face in addition to \option{smallfamily}
% \end{options}
%
% The package also provides a way to only change the text fonts or only the math fonts.
% \begin{options}
% \item[\option{onlytext}] only change the text fonts
% \item[\option{onlymath}] only change the math fonts
% \item[\option{sansmath}] provide mathversion \emph{sans} and
%   \emph{sansbold} independently of options \option{onlytext} and
%   \option{onlymath} and change |\mathsf| to use MyriadPro. This can
%   be used together with \option{onlytext} to only use MyriadPro's
%   math in a part of the document. Works only with \pkg{MinionPro} as
%   the main font at the moment.
% \end{options}
%
% \subsubsection*{Figure selection}
%
% MyriadPro offers four different figure versions. A detailed
% description is given in Section~\ref{sec:fig}.
% The default version can be selected by the following options:
%
% \begin{options}
% \item[\option{textosf}]     use text figures in text mode
% \item[\option{mathosf}]     use text figures in math mode
% \item[\option*{osf}]        use text figures in text and math mode
% \newblock
% \item[\option{textlf}]      use lining figures in text mode
% \item[\option{mathlf}]      use lining figures in math mode
% \item[\option{lf}]          use lining figures in text and math mode
% \newblock
% \item[\option{mathtabular}] use tabular figures in math mode
% \end{options}
%
% \subsubsection*{Calligraphic fonts}
%
% These options specify which font is used by the |\mathcal| command.
%
% \begin{options}
% \item[\option*{cmsy}]  take the calligraphic symbols from Computer Modern:
%                       {\fontencoding{OMS}\fontfamily{cmsy}\selectfont ABC}
% \item[\option{abx}]   use the calligraphic symbols provided by mathabx:
%                       \IfFileExists{mathabx.sty}{{\font\testabx=mathc10\testabx ABCabc}}{}\par
%                       (This font contains also lowercase letters, but it is not quite finished.)
% \item[\option{crswash[=\emph{option}]}] %
%                       use the swash letters from
%                       CronosPro. \emph{option} can be either
%                       \emph{noptsmall}, \emph{optsmall},
%                       \emph{noptmed} or \emph{optmed} using (no)
%                       optical weights, small or medium family
%                       configuration (see CronosPro
%                       documentation). First one is default.
% \end{options}
%
% \subsubsection*{Blackboard bold letters}
%
% You can also select different fonts for the |\mathbb| command.
%
% \begin{options}
% \item[\option*{amsbb}]    use the AMS blackboard font: $\mathbb{NZQRC}$
% \item[\option{fourierbb}] use the Fourier blackboard font:
%                           \IfFileExists{fourier.sty}{{\font\testfour=fourier-bb at 9.5pt\testfour NZQRC}}{}
% \item[\option{lucidabb}]  use the (commercial) Lucida Math blackboard font\relax
%\iffalse%
%                        : {\font\testluc=hlcra at 9.2pt\testluc NZQRC}%
%\fi%
% \end{options}
%
% \subsubsection*{Greek letters}
%
% The following options specify whether you want to use upright or italic Greek
% letters in math mode.
%
% \begin{options}
% \item[\option*{mixedgreek}] uppercase Greek is upright, lowercase
%                             Greek is italic
% \item[\option{italicgreek}] all Greek letters are italic
% \item[\option{frenchmath}]  all Greek letters and the uppercase Roman
%                             letters are upright
% \end{options}
%
% \noindent
% Upright and italic Greek letters are also directly accessible via the commands
% |\upgamma|, |\itgamma|, |\upGamma|, |\itGamma|, etc.
%
% \subsubsection*{Miscellaneous options}
%
% \begin{options}
% \item[\option{scaled=\emph{factor}}]
%      scale the font size by <\emph{factor}>
% \item[\option{myriadint}]
%      take the integral symbols from MyriadPro, not from
%      MdSymbol: $\displaystyle\int$ instead of $\displaystyle\varint$
% \item[\option{loosequotes}]
%      The quote signs of MyriadPro are set rather tight. This can
%      lead to undesirable spacing for apostrophes. The \option{loosequotes}
%      option slightly increases the side bearings of quotes.
%
%      This option requires \pdfTeX~1.40 and \pkg{microtype}~2.0.
%      Beware that this option prevents hyphenation of words
%      containing apostrophes. Such words will require explicit
%      hyphenation commands |\-|.
% \item[\option{footnotefigures}]
%      use special figures for footnote marks, i.e.,\newline
%      example\begingroup\fontfamily{MyriadPro-Extra}\fontencoding{U}\selectfont
%        6,9\endgroup\ instead of example\textsuperscript{\figureversion{lf}6,9}.
%
%      This option can only be used if the footnote marks consist
%      \emph{solely} of figures.
% \end{options}
%
% \section{Figure selection}
% \label{sec:fig}
%
% MyriadPro offers four different figure versions. One can choose between
% \emph{text figures} (lowercase figures) and \emph{lining figures} (uppercase figures)
% and one can choose between \emph{proportional} figures (figures with different widths)
% and \emph{tabular} figures (all figures have the same width, useful mainly for tables).
%
% \medskip\tabindent
% \begin{tabular}{@{}lll@{}}
% \toprule
%                & text figures & lining figures \\
% \midrule
%   proportional & \figureversion{text,proportional}0123456789
%                & \figureversion{lining,proportional}0123456789 \\
%   tabular      & \figureversion{text,tabular}0123456789
%                & \figureversion{lining,tabular}0123456789 \\
% \bottomrule
% \end{tabular}
%
% \medskip\noindent
% The |\figureversion| command can be used to switch between
% different figure versions. Possible parameters are:
%
% \begin{options}
% \item[\option{text}, \option{osf}]          text figures
% \item[\option{lining}, \option{lf}]         lining figures
% \item[\option{tabular}, \option{tab}]       tabular figures
% \item[\option{proportional}, \option{prop}] proportional figures
% \end{options}
%
% Usually it is desirable to set most text with proportional figures
% and to use tabular figures only in tables and lists. Unfortunately
% most \LaTeX\ document classes do not support fonts with several
% figure versions. Use the package \pkg{tabfigures} that patches some
% common document classes and packages (the standard \LaTeX\ classes,
% KOMA-Script, \pkg{memoir}, and \pkg{amsmath}) to use tabular figures
% at some places.
%
% \section{Additional font shapes and symbols}
%
% The \pkg{MyriadPro} package provides all symbols from the \pkg{MdSymbol} package.
% Additionally, the following math symbols are available:
%
% \medskip\tabindent
% \begingroup
% \newcommand*\sym[1]{$#1$ & \ttfamily\string#1}%
% \begin{tabular}{@{}*2{ll@{\qquad}}ll@{}}
%   \sym\openg       & \sym\varkappa       & \sym\varbeta \\
%   \sym\backepsilon & \sym\varbackepsilon & \sym\hbar \\
%   \sym\hslash      & \sym\lambdabar      & \sym\lambdaslash \\
%   \sym\jmath       & \sym\eth            & \sym\Bbbk \\
%   \sym\slashedzero  \\
% \end{tabular}
% \endgroup
% Some of the alternative characters above resemble the normal
% character because MyriadPro offers no respective glyph.  They are
% defined for compatibility reasons.
%
% \medskip\noindent
% Small and slanted fractions are fractions with a height matching the font's
% body size. These are useful for typesetting, e.g.,
% $\cos(\smallfrac{1}{2} x + \smallfrac{3}{2} y)$ or ``\slantfrac{1}{12}~litres of
% red wine'' and can be accessed via
%
% \medskip\tabindent
% \begin{tabular}{@{}ll@{ }l}
%   |\smallfrac|\marg{numerator}\marg{denominator}& \smallfrac{1}{3} & \smallfrac{5}{17}\\
%   |\slantfrac|\marg{numerator}\marg{denominator}& \slantfrac{1}{3} & \slantfrac{5}{17}
% \end{tabular}
%
% \medskip\noindent
% Note that \emph{only} figures can be used for \meta{numerator} and \meta{denominator}.
%
% \section{Language support}
%
% The following encodings are supported:
%
% \medskip\tabindent
% \begin{tabular}{@{}ll@{}}
%   Latin    & \acro{OT1,} \acro{T1,} \acro{TS1,} \acro{LY1,} \acro{T5}\\
%   Cyrillic & \acro{T2A,} \acro{T2B,} \acro{T2C,} \acro{X2,} \acro{OT2}\\
%   Greek    & \acro{LGR} (to be used with \pkg{babel}, including \option{polutonikogreek}),\\
%            & \acro{LGI} (\pkg{Ibycus} transliteration scheme)
% \end{tabular}
%
% \medskip\noindent
% In order to typeset Greek text with the \pkg{Ibycus} transliteration scheme, specify
% \begin{quote}
%   |\usepackage[ibycus,|\meta{otherlanguages}|]{babel}|
% \end{quote}
% in the preamble and consult the documentation given in |ibycus-babel.pdf|
% on \acro{CTAN}. |\setgreekfontsize| is not supported.
%
% \section{Searching for figures or for words containing ligatures in \acro{PDF} documents}
%
% Searching for figures or for words containing ligatures in \acro{PDF} documents may not
% be possible depending on the way the \acro{PDF} file was created. The following table
% gives an overview of which glyphs may cause problems.
%
% \medskip
% \begin{tabular}{>{\tbfigures}lll}
% \toprule
%   font version & program        & problems \\
% \midrule
%   1.000  & Ghostscript,         & LF/TOsF, non-standard ligatures\\
%          & pre-1.40 \pdfTeX     &\\
% \midrule
%   1.001, & Ghostscript,         & LF/OsF/TOsF, ligatures\\
%   2.000  & pre-1.40 \pdfTeX     & \\
% \midrule
%   1.00x  & Distiller, dvipdfmx  & LF/TOsF\\
% \midrule
%   1.00x  & \pdfTeX~1.40         & ok\\
% \midrule
%   2.000  & Distiller, dvipdfmx, & ok\\
%          & \pdfTeX~1.40         & \\
% \bottomrule
% \end{tabular}
%
% \medskip\noindent
% To make figures and ligatures searchable when using \pdfTeX~1.40,
% you need to enable glyph-to-unicode translation and load the default
% mapping table:
% \begin{verbatim}
% \input glyphtounicode
% \pdfgentounicode=1
% \end{verbatim}
% See the \pdfTeX\ manual for details.
%
% \section{\acro{NFSS} classification}
%
% Parenthesised combinations are provided via substitutions.
%
% \nobreak\medskip\noindent
% \begingroup
% \centering\let\>=\\%
% \begin{tabular}{@{}p{7.5em}p{7em}lp{9em}@{}}
%   \toprule
%   encoding & family & series & shape \\
%   \midrule
%   \raggedright \acro{OT1,} \acro{T1,} \acro{TS1,} \acro{LY1,} \acro{T5} &
%   MyriadPro-OsF, MyriadPro-LF, MyriadPro-TOsF, MyriadPro-TLF &
%   m, b (sb, bx), eb &
%   n, it (sl)\\
%   \addlinespace
%   \midrule
%   \addlinespace
%   \raggedright \acro{LGR,} \acro{LGI,} \acro{T2A,} \acro{T2B,} \acro{T2C,} \acro{X2,} \acro{OT2} &
%   MyriadPro-OsF, MyriadPro-LF, MyriadPro-TOsF, MyriadPro-TLF &
%   m, b (sb, bx), eb &
%   n, it (sl) \\
%   \addlinespace
%   \midrule
%   \addlinespace
%   \acro{OML} &
%   MyriadPro-TOsF &
%   m, b (sb, bx), eb &
%   n, it \\
%   \addlinespace
%   \midrule
%   \addlinespace
%   \acro{U} &
%   MyriadPro-Extra &
%   m, b (sb, bx), eb &
%   n, it (sl) \\
%   \bottomrule
% \end{tabular}\par
% \endgroup
%
% \section{Version history}
%
% Version 0.1alpha: Development version
%
% \StopEventually{}
%
% \section{The main style file}
%
% \subsection{Options}
%
%    \begin{macrocode}
%<*style>
\newif\if@My@Text@
\newif\if@My@Math@
\newif\if@My@Sans@Math@
\@My@Text@true
\@My@Math@true
\@My@Sans@Math@false
\RequirePackage{kvoptions}
\SetupKeyvalOptions{
  family = My,
  prefix = My@
}
\DeclareVoidOption{onlytext}{\@My@Text@true\@My@Math@false}
\DeclareVoidOption{onlymath}{\@My@Text@false\@My@Math@true}
\DeclareVoidOption{sansmath}{\@My@Sans@Math@true}
%    \end{macrocode}
%
% \subsubsection{Font sets}
%
% The package \pkg{MyriadPro-FontDef} adapts the font definitions to
% the requested font set (see section~\ref{sec:fontdef}). So we simply
% pass on the relevant options including the font scale factor; only
% MyriadPro integrals are handled here in \pkg{MyriadPro}.
%    \begin{macrocode}
\DeclareStringOption[1.]{scaled}
\newcommand\My@myriadint@opticals{-NoOpticals}
\newcommand\My@myriadint@bold{-Bold}
\newcommand\My@mdsym@regular{regular}
\DeclareVoidOption{noopticals}{%
  \def\My@myriadint@opticals{-NoOpticals}%
  \PassOptionsToPackage{noopticals}{MyriadPro-FontDef}}
\DeclareVoidOption{smallfamily}{%
  \def\My@myriadint@bold{-Bold}%
  \PassOptionsToPackage{smallfamily}{MyriadPro-FontDef}}
\DeclareVoidOption{medfamily}{%
  \def\My@myriadint@bold{-Semibold}%
  \def\My@mdsym@regular{autoregular}%
  \PassOptionsToPackage{medfamily}{MyriadPro-FontDef}}
%\DeclareVoidOption{fullfamily}{%
%  \def\My@myriadint@bold{-Semibold}%
%  \PassOptionsToPackage{fullfamily}{MyriadPro-FontDef}}
\DeclareVoidOption{normalsize}{%
  \PassOptionsToPackage{normalsize}{MyriadPro-FontDef}}
%    \end{macrocode}
%
% \subsubsection{Figure style}
%
%    \begin{macrocode}
\newcommand\My@Text@Fig{OsF}
\newcommand\My@Math@Fig{OsF}
\newcommand\My@Text@Family{MyriadPro-\My@Text@Fig}
\newcommand\My@Math@Family{MyriadPro-\My@Math@Fig}
\newcommand\My@Math@TFamily{MyriadPro-T\My@Math@Fig}
\newcommand\My@Math@LetterShape{it}
\newcommand\Cr@Math@Family{CronosPro-\My@Math@Fig}
\newcommand\Cr@Math@TFamily{CronosPro-T\My@Math@Fig}
%    \end{macrocode}
%
%    \begin{macrocode}
\DeclareVoidOption{textosf}{\def\My@Text@Fig{OsF}}
\DeclareVoidOption{textlf}{\def\My@Text@Fig{LF}}
\DeclareVoidOption{mathosf}{\def\My@Math@Fig{OsF}}
\DeclareVoidOption{mathlf}{\def\My@Math@Fig{LF}}
\DeclareVoidOption{osf}{\setkeys{My}{textosf,mathosf}}
\DeclareVoidOption{lf}{\setkeys{My}{textlf,mathlf}}
\DeclareVoidOption{mathtabular}{\let\My@Math@Family\My@Math@TFamily}
%    \end{macrocode}
%
% \subsubsection{Calligraphic fonts}
%
% These hooks are executed once the math versions have been set up.
%    \begin{macrocode}
\newcommand\My@load@cal{}
\newcommand\My@load@sans@cal{}
\newcommand\My@load@bb{}
\newcommand\My@load@frak{}
%    \end{macrocode}
% Most options are handled by \pkg{MdSymbol}.
%    \begin{macrocode}
\DeclareStringOption[false]{crswash}[noptsmall]
%    \end{macrocode}
%
% \subsubsection{Greek letters}
%
% |\My@greek@Upright|, |\My@greek@Mixed|, and |\My@greek@Italic| are defined below in
% section~\ref{sec:greek} before |\My@load@greek| is executed.
%    \begin{macrocode}
\newcommand\My@load@greek{\My@greek@Mixed}
\DeclareVoidOption{frenchmath}{%
  \def\My@load@greek{\My@greek@Upright}%
  \def\My@Math@LetterShape{n}}
\DeclareVoidOption{mixedgreek}{%
  \def\My@load@greek{\My@greek@Mixed}}
\DeclareVoidOption{italicgreek}{%
  \def\My@load@greek{\My@greek@Italic}}
%    \end{macrocode}
%
% \subsubsection{Blackboard bold and fraktur fonts}
%
% We have to undefine |\mathfrak| and |\mathbb| before redefining them, because
% they might be defined in such a way that |\DeclareMathAlphabet| does not
% recognize them as math alphabets and refuses to overwrite their definitions
% (e.g., package \pkg{eufrak} uses |\newcommand{\mathfrak}{\EuFrak}|).
%    \begin{macrocode}
\newcommand\My@load@amsbb{
  \let\mathbb\@undefined
  \let\Bbbk\@undefined
  \DeclareMathAlphabet\mathbb{U}{msb}{m}{n}
  \newcommand\Bbbk{\mathbb{\mathchar"717C}}}
\newcommand\My@load@lucidabb{
  \let\mathbb\@undefined
  \let\Bbbk\@undefined
  \DeclareFontFamily{U}{hlcm}{}
  \DeclareFontShape{U}{hlcm}{m}{n}{ <->s*[0.92] hlcra }{}
  \DeclareMathAlphabet\mathbb{U}{hlcm}{m}{n}
  \newcommand\Bbbk{\mathbb{k}}}
\newcommand\My@load@fourierbb{
  \let\mathbb\@undefined
  \let\Bbbk\@undefined
  \DeclareFontFamily{U}{futm}{}
  \DeclareFontShape{U}{futm}{m}{n}{ <->s*[0.95] fourier-bb }{}
  \DeclareMathAlphabet\mathbb{U}{futm}{m}{n}
  \newcommand\Bbbk{\mathbb{k}}}
\DeclareVoidOption{amsbb}{\let\My@load@bb\My@load@amsbb}
\DeclareVoidOption{lucidabb}{\let\My@load@bb\My@load@lucidabb}
\DeclareVoidOption{fourierbb}{\let\My@load@bb\My@load@fourierbb}
%    \end{macrocode}
%
% \subsubsection{Integrals}
%
%    \begin{macrocode}
\newcommand\My@load@integrals{}
\DeclareVoidOption{myriadint}{\def\My@load@integrals{\My@Decl@Myriad@Ints}}
%    \end{macrocode}
%
% \subsubsection{Miscellaneous options}
%
% Footnote figures, the g glyph in math mode, extra spacing for the apostrophe.
%    \begin{macrocode}
\DeclareVoidOption{footnotefigures}{%
  \def\@makefnmark{%
    \begingroup
    \normalfont
    \fontfamily{MyriadPro-Extra}\fontencoding{U}\selectfont
    \@thefnmark
    \endgroup}}
%
\newcommand\My@Define@Open@g{}
\DeclareVoidOption{openg}{%
  \def\My@Define@Open@g{%
    \mathcode`g="8000%
    \DeclareMathSymbol{\My@g}{\mathalpha}{letters}{`g}%
    \begingroup
    \lccode`\~=`\g
    \lowercase{\gdef~{\ifnum\the\mathgroup=\m@ne \openg \else \My@g \fi}}%
    \endgroup
  }}
%
\newcommand\My@Quote@Spacing{}
\DeclareVoidOption{loosequotes}{%
  \def\My@Quote@Spacing{\My@Quote@Spacing@Loose}}
%    \end{macrocode}
%
% \subsubsection{Defaults}
%
%    \begin{macrocode}
\setkeys{My}{amsbb}
\ProcessKeyvalOptions{My}\relax
\RequirePackage{ifthen}
\def\My@setup@crswash{
    \RequirePackage{fltpoint}
    \fpDecimalSign{.}
    \fpMul{\Cr@scaled}{\My@scaled}{1.08}
    \ifthenelse{\equal{\My@crswash}{noptsmall}}{%
      \RequirePackage{CronosPro-FontDef}}{}
    \ifthenelse{\equal{\My@crswash}{optsmall}}{%
      \RequirePackage[opticals]{CronosPro-FontDef}}{}
    \ifthenelse{\equal{\My@crswash}{noptmed}}{%
      \RequirePackage[medfamily]{CronosPro-FontDef}}{}
    \ifthenelse{\equal{\My@crswash}{optmed}}{%
      \RequirePackage[opticals,medfamily]{CronosPro-FontDef}}{}}
\ifthenelse{\equal{\My@crswash}{false}}{}{%
  \def\My@load@cal{
    \My@setup@crswash
    \DeclareMathAlphabet\mathcal         {T1}{\Cr@Math@Family} {m} {sw}
    \SetMathAlphabet\mathcal{bold}       {T1}{\Cr@Math@Family} {eb}{sw}
    \SetMathAlphabet\mathcal{tabular}    {T1}{\Cr@Math@TFamily}{m} {sw}
    \SetMathAlphabet\mathcal{boldtabular}{T1}{\Cr@Math@TFamily}{eb}{sw}}
  \def\My@load@sans@cal{
    \My@setup@crswash
    \@ifundefined{mathfrak}{%
      \DeclareMathAlphabet\mathcal    {T1}{\Cr@Math@Family}{m} {sw}}
    \SetMathAlphabet\mathcal{sans}    {T1}{\Cr@Math@Family}{m} {sw}
    \SetMathAlphabet\mathcal{sansbold}{T1}{\Cr@Math@Family}{eb}{sw}}}
%    \end{macrocode}
%
% \subsection{Font declarations}
%
%    \begin{macrocode}
\RequirePackage{MyriadPro-FontDef}
\@ifpackageloaded{textcomp}{}{\RequirePackage{textcomp}}

\if@My@Math@
  \RequirePackage[normalweight=\My@mdsym@regular,boldweight=bold,scale=\My@scaled]{mdsymbol}
%    \end{macrocode}
% If no fraktur font is loaded then take the Euler font.
%    \begin{macrocode}
  \@ifundefined{mathfrak}{%
    \RequirePackage{eufrak}%
    \SetMathAlphabet\EuFrak{boldtabular}{U}{euf}{b}{n}}{}
\fi
%    \end{macrocode}
% By default, we use |b| for the bold series. If |MyriadPro-Semibold| is not
% available this might internally be mapped to |MyriadPro-Bold| (see
% \pkg{MyriadPro-FontDef}).
%    \begin{macrocode}
\if@My@Text@
  \edef\sfdefault{\My@Text@Family}
  \let\ibycusdefault\My@Text@Family
%    \end{macrocode}
% If a recent verion of \pkg{microtype} is loaded then we implement an option
% to increase the side bearings of all quote glyphs.
%    \begin{macrocode}
  \def\My@Quote@Spacing@Loose{%
    \@ifpackageloaded{microtype}{}{\RequirePackage[kerning=true]{microtype}}
    \@ifundefined{SetExtraKerning}{}{
      \let\My@Set@Quote@Spacing\SetExtraKerning}
%      \SetExtraKerning
%        [ unit = 1em ]
%        { encoding = {OT1,T1,LGR,U,OT2,T2A,T2B,T2C,T5,X2,LY1},
%          family   = {MyriadPro-OsF,MyriadPro-LF,MyriadPro-TOsF,MyriadPro-TLF},
%          shape    = n }
%        { \textquotedblleft = {30,30},  \textquotedblright = {30,30},
%          \textquoteleft    = {30,30},  \textquoteright    = {30,30} }}
  }
  \newcommand*\My@Set@Quote@Spacing[3][]{}
  \My@Quote@Spacing
  \My@Set@Quote@Spacing
    [ unit = 1em ]
    { encoding = {OT1,T1,LGR,U,OT2,T2A,T2B,T2C,T5,X2,LY1},
      family   = {MyriadPro-OsF,MyriadPro-LF,MyriadPro-TOsF,MyriadPro-TLF},
      shape    = {n,it} }
    { \textquotedblleft = {30,30},  \textquotedblright = {30,30},
      \textquoteleft    = {30,30},  \textquoteright    = {30,30} }
\fi
%    \end{macrocode}
%
% \subsubsection{Math fonts}
%
% Redefine the standard math versions |normal| and |bold|.
%    \begin{macrocode}
\if@My@Math@
  \DeclareSymbolFont{operators}  {T1} {\My@Math@Family}{m} {n}
  \DeclareSymbolFont{letters}    {OML}{MyriadPro-TOsF} {m} {\My@Math@LetterShape}
  \SetSymbolFont{operators}{bold}{T1} {\My@Math@Family}{eb}{n}
  \SetSymbolFont{letters}  {bold}{OML}{MyriadPro-TOsF} {eb}{\My@Math@LetterShape}
  \DeclareMathAlphabet\mathbf    {T1} {\My@Math@Family}{eb}{n}
  \DeclareMathAlphabet\mathsf    {T1} {\My@Math@Family}{m} {n}
  \SetMathAlphabet\mathsf  {bold}{T1} {\My@Math@Family}{eb}{n}
  \DeclareMathAlphabet\mathit    {T1} {\My@Math@Family}{m} {it}
  \SetMathAlphabet\mathit  {bold}{T1} {\My@Math@Family}{eb}{it}
%    \end{macrocode}
% Extra math versions |tabular| and |boldtabular|, which use tabular figures
% instead of proportional ones. These math versions can be useful in tables
% (cf.\ section~\ref{sec:Interferences}).
%    \begin{macrocode}
  \DeclareMathVersion{tabular}
  \SetSymbolFont{operators}{tabular}    {T1} {\My@Math@TFamily}{m}{n}
  \SetSymbolFont{letters}  {tabular}    {OML}{MyriadPro-TOsF}  {m}{\My@Math@LetterShape}
  \SetMathAlphabet\mathit  {tabular}    {T1} {\My@Math@TFamily}{m}{it}

  \DeclareMathVersion{boldtabular}
  \SetSymbolFont{operators}{boldtabular}{T1} {\My@Math@TFamily}{eb}{n}
  \SetSymbolFont{letters}  {boldtabular}{OML}{MyriadPro-TOsF}  {eb}{\My@Math@LetterShape}
  \SetMathAlphabet\mathit  {boldtabular}{T1} {\My@Math@TFamily}{eb}{it}
%    \end{macrocode}
%
%    \begin{macrocode}
  \DeclareMathAccent{\grave}   {\mathalpha}{operators}{0}
  \DeclareMathAccent{\acute}   {\mathalpha}{operators}{1}
  \DeclareMathAccent{\hat}     {\mathalpha}{operators}{2}
  \DeclareMathAccent{\tilde}   {\mathalpha}{operators}{3}
  \DeclareMathAccent{\ddot}    {\mathalpha}{operators}{4}
  \DeclareMathAccent{\mathring}{\mathalpha}{operators}{6}
  \DeclareMathAccent{\check}   {\mathalpha}{operators}{7}
  \DeclareMathAccent{\breve}   {\mathalpha}{operators}{8}
  \DeclareMathAccent{\bar}     {\mathalpha}{operators}{9}
  \DeclareMathAccent{\dot}     {\mathalpha}{operators}{10}
%    \end{macrocode}
% Execute the hooks set up above to load the various math alphabets.
%    \begin{macrocode}
  \My@load@bb
  \My@load@frak
  \My@load@cal
\fi
%    \end{macrocode}
% Setup for sans serif math: set mathsf, create two new math versions
% for sans serif math and load correct swash letters.
%    \begin{macrocode}
\if@My@Sans@Math@

  \newcommand\IfSymbolFont[3]{\@ifundefined{sym#1}{#3}{#2}}

  \DeclareMathAlphabet\mathsf    {T1} {\My@Math@Family}{m} {n}
  \SetMathAlphabet\mathsf  {bold}{T1} {\My@Math@Family}{eb}{n}

  \DeclareMathVersion{sans}
  \SetMathAlphabet\mathit  {sans}{T1}{\My@Math@Family}{m}{it}
  \IfSymbolFont{operators}{%
    \SetSymbolFont{operators}{sans}{T1} {\My@Math@Family}{m}{n}
  }{%
    \DeclareSymbolFont{operators}  {T1} {\My@Math@Family}{m}{n}
  }
  \IfSymbolFont{letters}{%
    \SetSymbolFont{letters}{sans}{OML}{MyriadPro-OsF}{r}{\My@Math@LetterShape}
  }{%
    \DeclareSymbolFont{letters}  {OML}{MyriadPro-OsF}{r}{\My@Math@LetterShape}
  }
  \IfSymbolFont{MnSyA}{%
    \SetSymbolFont{MnSyA}    {sans}{U}  {MnSymbolA}{m}{n}
  }{%
    \DeclareSymbolFont{MnSyA}{sans}{U}  {MnSymbolA}{m}{n}
  }
  \IfSymbolFont{MnSyB}{%
    \SetSymbolFont{MnSyB}    {sans}{U}  {MnSymbolB}{m}{n}
  }{%
    \DeclareSymbolFont{MnSyB}{sans}{U}  {MnSymbolB}{m}{n}
  }
  \IfSymbolFont{MnSyC}{%
    \SetSymbolFont{MnSyC}    {sans}{U}  {MnSymbolC}{m}{n}
  }{%
    \DeclareSymbolFont{MnSyC}{sans}{U}  {MnSymbolC}{m}{n}
  }
  \IfSymbolFont{MnSyD}{%
    \SetSymbolFont{MnSyD}    {sans}{U}  {MnSymbolD}{m}{n}
  }{%
    \DeclareSymbolFont{MnSyD}{sans}{U}  {MnSymbolD}{m}{n}
  }
  \IfSymbolFont{largesymbols}{%
    \SetSymbolFont{largesymbols}{sans}{OMX}{MnSymbolE}{m}{n}
  }{%
    \DeclareSymbolFont{largesymbols}   {OMX}{MnSymbolE}{m}{n}
  }
  \IfSymbolFont{symbols}{%
    \SetSymbolFont{symbols}  {sans}{U}  {MnSymbolF}{m}{n}
  }{%
    \DeclareSymbolFont{symbols}        {U}  {MnSymbolF}{m}{n}
  }


  \DeclareMathVersion{sansbold}
  \SetMathAlphabet\mathit  {sansbold}{T1}{\My@Math@Family}{eb}{it}
  \SetSymbolFont{operators}{sansbold}{T1}{\My@Math@Family}{eb}{n}
  \SetSymbolFont{letters}  {sansbold}{OML}{MyriadPro-OsF}
  {eb}{\My@Math@LetterShape}
  \SetSymbolFont{MnSyA}       {sansbold}{U}  {MnSymbolA}{b}{n}
  \SetSymbolFont{MnSyB}       {sansbold}{U}  {MnSymbolB}{b}{n}
  \SetSymbolFont{MnSyC}       {sansbold}{U}  {MnSymbolC}{b}{n}
  \SetSymbolFont{MnSyD}       {sansbold}{U}  {MnSymbolD}{b}{n}
  \SetSymbolFont{largesymbols}{sansbold}{OMX}{MnSymbolE}{b}{n}
  \SetSymbolFont{symbols}     {sansbold}{U}  {MnSymbolF}{b}{n}

  \My@load@sans@cal
\fi
%    \end{macrocode}
%
% \subsection{Font selection}
%
% The font selection commands such as |\figureversion| are provided by
% the package \pkg{fontaxes}.
%
%    \begin{macrocode}
\RequirePackage{fontaxes}[2005/05/04]
%    \end{macrocode}
% We define an additional short hand for compatibility's sake.
%    \begin{macrocode}
\let\oldstylenums\textfigures
%    \end{macrocode}
%
% \subsection{Greek letters}
% \label{sec:greek}
%
% We provide math-mode commands for each Greek letter, both italic and upright.
% Furthermore, there are three commands to select the default version of the
% letters (all upright, all italic, or capitals upright and lowercase italic).
%
% While declaring the Greek letters we collect the uppercase and lowercase
% letters in two lists. (We distinguish them by the first letter of their
% name.) These lists are then used to select the different versions.
%    \begin{macrocode}
\if@My@Math@
  \newcommand\My@greek@list@upper{}
  \newcommand\My@greek@list@lower{}
  \let\My@greek@list@upper\@gobble
  \let\My@greek@list@lower\@gobble
%    \end{macrocode}
% This macro holds one of the two list names.
%    \begin{macrocode}
  \newcommand\My@greek@list{}
  \newcommand*\My@greek@letter[3]{%
    \expandafter\DeclareMathSymbol
    \expandafter{\csname it#1\endcsname}{\mathord}{letters}{#2}%
    \expandafter\DeclareMathSymbol
    \expandafter{\csname up#1\endcsname}{\mathord}{letters}{#3}%
    \edef\@tempa{`\@car#1\@nil}%
    \edef\My@greek@list{\expandafter\noexpand\csname
      My@greek@list@\ifnum\uccode\@tempa=\@tempa upper\else lower\fi\endcsname}%
    \expandafter\edef\My@greek@list{\My@greek@list,#1}%
  }
%    \end{macrocode}
% We can now declare the Greek letters (left italic, right upright).
%    \begin{macrocode}
  \My@greek@letter{Gamma}         {'000}{'200}
  \My@greek@letter{Delta}         {'001}{'201}
  \My@greek@letter{Theta}         {'002}{'202}
  \My@greek@letter{Lambda}        {'003}{'203}
  \My@greek@letter{Xi}            {'004}{'204}
  \My@greek@letter{Pi}            {'005}{'205}
  \My@greek@letter{Sigma}         {'006}{'206}
  \My@greek@letter{Upsilon}       {'007}{'207}
  \My@greek@letter{Phi}           {'010}{'210}
  \My@greek@letter{Psi}           {'011}{'211}
  \My@greek@letter{Omega}         {'012}{'212}
  \My@greek@letter{alpha}         {'013}{'213}
  \My@greek@letter{beta}          {'014}{'214}
  \My@greek@letter{gamma}         {'015}{'215}
  \My@greek@letter{delta}         {'016}{'216}
  \My@greek@letter{epsilon}       {'017}{'217}
  \My@greek@letter{zeta}          {'020}{'220}
  \My@greek@letter{eta}           {'021}{'221}
  \My@greek@letter{theta}         {'022}{'222}
  \My@greek@letter{iota}          {'023}{'223}
  \My@greek@letter{kappa}         {'024}{'224}
  \My@greek@letter{lambda}        {'025}{'225}
  \My@greek@letter{mu}            {'026}{'226}
  \My@greek@letter{nu}            {'027}{'227}
  \My@greek@letter{xi}            {'030}{'230}
  \My@greek@letter{pi}            {'031}{'231}
  \My@greek@letter{rho}           {'032}{'232}
  \My@greek@letter{sigma}         {'033}{'233}
  \My@greek@letter{tau}           {'034}{'234}
  \My@greek@letter{upsilon}       {'035}{'235}
  \My@greek@letter{phi}           {'036}{'236}
  \My@greek@letter{chi}           {'037}{'237}
  \My@greek@letter{psi}           {'040}{'240}
  \My@greek@letter{omega}         {'041}{'241}
  \My@greek@letter{varepsilon}    {'042}{'242}
  \My@greek@letter{vartheta}      {'043}{'243}
  \My@greek@letter{varpi}         {'044}{'244}
  \My@greek@letter{varrho}        {'045}{'245}
  \My@greek@letter{varsigma}      {'046}{'246}
  \My@greek@letter{varphi}        {'047}{'247}
%    \end{macrocode}
% Some of the following symbols are not really Greek letters but are treated in
% the same way.
%    \begin{macrocode}
%%  \My@greek@letter{varbeta}       {'260}{'250}
  \My@greek@letter{varbeta}       {'014}{'214}
%%  \My@greek@letter{varkappa}      {'261}{'251}
  \My@greek@letter{varkappa}      {'024}{'224}
  \My@greek@letter{backepsilon}   {'262}{'252}
  \My@greek@letter{varbackepsilon}{'263}{'253}
  \My@greek@letter{digamma}       {'264}{'254}
  \My@greek@letter{eth}           {'266}{'256}
%    \end{macrocode}
% Go through a list |#2| of Greek letters and |\let| them be their
% |#1|-prefixed variants.
%    \begin{macrocode}
  \newcommand*\My@greek@select[2]{%
    \expandafter\let\expandafter\My@greek@list\csname My@greek@list@#2\endcsname
    \@for\@tempa:=\My@greek@list\do{%
      \expandafter\let\csname\@tempa\expandafter\endcsname
      \csname#1\@tempa\endcsname
    }%
  }
  \newcommand*\My@greek@Upright{%
    \My@greek@select{up}{upper}%
    \My@greek@select{up}{lower}%
  }
  \newcommand*\My@greek@Italic{%
    \My@greek@select{it}{upper}%
    \My@greek@select{it}{lower}%
  }
  \newcommand*\My@greek@Mixed{%
    \My@greek@select{up}{upper}%
    \My@greek@select{it}{lower}%
  }
%    \end{macrocode}
% Finally initialise the Greek letters.
%    \begin{macrocode}
  \My@load@greek
\fi
%    \end{macrocode}
%
% \subsection{\pdfTeX\ to-unicode support}
%
% Old versions of MyriadPro have non-standard glyph names.
%    \begin{macrocode}
\@ifundefined{pdfglyphtounicode}{}{
  \pdfglyphtounicode{uniEFD5}{03DD}% uni03DD
  \pdfglyphtounicode{uniEFED}{02D9}% dotaccent.cap
  \pdfglyphtounicode{uniEFEE}{02D8}% breve.cap
  \pdfglyphtounicode{uniEFF1}{02DB}% ogonek.cap
  \pdfglyphtounicode{uniEFF2}{00B8}% cedilla.cap
  \pdfglyphtounicode{uniEFF3}{02DA}% ring.cap
  \pdfglyphtounicode{uniEFF5}{02DC}% tilde.cap
  \pdfglyphtounicode{uniEFF7}{02C6}% circumflex.cap
  \pdfglyphtounicode{uniF628}{2030}% perthousand.oldstyle
  \pdfglyphtounicode{uniF62C}{0028}% parenleft.denominator
  \pdfglyphtounicode{uniF62D}{0029}% parenright.denominator
  \pdfglyphtounicode{uniF631}{0028}% parenleft.numerator
  \pdfglyphtounicode{uniF632}{0029}% parenright.numerator
  \pdfglyphtounicode{uniF638}{0030}% zero.slash
  \pdfglyphtounicode{uniF639}{0030}% zero.fitted
  \pdfglyphtounicode{uniF63A}{0032}% two.fitted
  \pdfglyphtounicode{uniF63B}{0033}% three.fitted
  \pdfglyphtounicode{uniF63C}{0034}% four.fitted
  \pdfglyphtounicode{uniF63D}{0035}% five.fitted
  \pdfglyphtounicode{uniF63E}{0036}% six.fitted
  \pdfglyphtounicode{uniF63F}{0037}% seven.fitted
  \pdfglyphtounicode{uniF640}{0038}% eight.fitted
  \pdfglyphtounicode{uniF641}{0039}% nine.fitted
  \pdfglyphtounicode{uniF642}{0025}% percent.oldstyle
  \pdfglyphtounicode{uniF643}{0030}% zero.taboldstyle
  \pdfglyphtounicode{uniF644}{0031}% one.taboldstyle
  \pdfglyphtounicode{uniF645}{0032}% two.taboldstyle
  \pdfglyphtounicode{uniF646}{0033}% three.taboldstyle
  \pdfglyphtounicode{uniF647}{0034}% four.taboldstyle
  \pdfglyphtounicode{uniF648}{0035}% five.taboldstyle
  \pdfglyphtounicode{uniF649}{0036}% six.taboldstyle
  \pdfglyphtounicode{uniF64A}{0037}% seven.taboldstyle
  \pdfglyphtounicode{uniF64B}{0038}% eight.taboldstyle
  \pdfglyphtounicode{uniF64C}{0039}% nine.taboldstyle
  \pdfglyphtounicode{uniF64D}{20A1}% colonmonetary.taboldstyle
  \pdfglyphtounicode{uniF64E}{20AC}% Euro.taboldstyle
  \pdfglyphtounicode{uniF64F}{0192}% florin.taboldstyle
  \pdfglyphtounicode{uniF650}{0023}% numbersign.taboldstyle
  \pdfglyphtounicode{uniF651}{00A3}% sterling.taboldstyle
  \pdfglyphtounicode{uniF652}{00A5}% yen.taboldstyle
  \pdfglyphtounicode{uniF653}{0024}% dollar.taboldstyle
  \pdfglyphtounicode{uniF654}{00A2}% cent.taboldstyle
  \pdfglyphtounicode{uniF655}{0030}% zero.denominator
  \pdfglyphtounicode{uniF656}{0031}% one.denominator
  \pdfglyphtounicode{uniF657}{0032}% two.denominator
  \pdfglyphtounicode{uniF658}{0033}% three.denominator
  \pdfglyphtounicode{uniF659}{0034}% four.denominator
  \pdfglyphtounicode{uniF65A}{0035}% five.denominator
  \pdfglyphtounicode{uniF65B}{0036}% six.denominator
  \pdfglyphtounicode{uniF65C}{0037}% seven.denominator
  \pdfglyphtounicode{uniF65D}{0038}% eight.denominator
  \pdfglyphtounicode{uniF65E}{0039}% nine.denominator
  \pdfglyphtounicode{uniF65F}{002C}% comma.denominator
  \pdfglyphtounicode{uniF660}{002E}% period.denominator
  \pdfglyphtounicode{uniF661}{0030}% zero.numerator
  \pdfglyphtounicode{uniF662}{0031}% one.numerator
  \pdfglyphtounicode{uniF663}{0032}% two.numerator
  \pdfglyphtounicode{uniF664}{0033}% three.numerator
  \pdfglyphtounicode{uniF665}{0034}% four.numerator
  \pdfglyphtounicode{uniF666}{0035}% five.numerator
  \pdfglyphtounicode{uniF667}{0036}% six.numerator
  \pdfglyphtounicode{uniF668}{0037}% seven.numerator
  \pdfglyphtounicode{uniF669}{0038}% eight.numerator
  \pdfglyphtounicode{uniF66A}{0039}% nine.numerator
  \pdfglyphtounicode{uniF66B}{002C}% comma.numerator
  \pdfglyphtounicode{uniF66C}{002E}% period.numerator
  \pdfglyphtounicode{uniF66D}{0103}% abreve.sc
  \pdfglyphtounicode{uniF66F}{0105}% aogonek.sc
  \pdfglyphtounicode{uniF671}{0107}% cacute.sc
  \pdfglyphtounicode{uniF672}{010D}% ccaron.sc
  \pdfglyphtounicode{uniF675}{010F}% dcaron.sc
  \pdfglyphtounicode{uniF676}{0111}% dcroat.sc
  \pdfglyphtounicode{uniF678}{011B}% ecaron.sc
  \pdfglyphtounicode{uniF67B}{014B}% eng.sc
  \pdfglyphtounicode{uniF67C}{0119}% eogonek.sc
  \pdfglyphtounicode{uniF67D}{011F}% gbreve.sc
  \pdfglyphtounicode{uniF684}{0133}% ij.sc
  \pdfglyphtounicode{uniF687}{0129}% itilde.sc
  \pdfglyphtounicode{uniF68A}{013A}% lacute.sc
  \pdfglyphtounicode{uniF68B}{013E}% lcaron.sc
  \pdfglyphtounicode{uniF68E}{0144}% nacute.sc
  \pdfglyphtounicode{uniF68F}{0148}% ncaron.sc
  \pdfglyphtounicode{uniF692}{0151}% ohungarumlaut.sc
  \pdfglyphtounicode{uniF695}{0155}% racute.sc
  \pdfglyphtounicode{uniF696}{0159}% rcaron.sc
  \pdfglyphtounicode{uniF698}{015B}% sacute.sc
  \pdfglyphtounicode{uniF699}{015F}% scedilla.sc
  \pdfglyphtounicode{uniF69D}{0165}% tcaron.sc
  \pdfglyphtounicode{uniF69E}{0163}% tcommaaccent.sc
  \pdfglyphtounicode{uniF6A0}{0171}% uhungarumlaut.sc
  \pdfglyphtounicode{uniF6A3}{016F}% uring.sc
  \pdfglyphtounicode{uniF6A4}{0169}% utilde.sc
  \pdfglyphtounicode{uniF6AA}{1EF3}% ygrave.sc
  \pdfglyphtounicode{uniF6AB}{017A}% zacute.sc
  \pdfglyphtounicode{uniF6AC}{017C}% zdotaccent.sc
  \pdfglyphtounicode{uniF6DC}{0031}% one.fitted
}
%    \end{macrocode}
%
% \subsection{Superior and inferior figures}
%
% We define commands to convert numbers to numerator figures and denominator figures.
%    \begin{macrocode}
\def\@for@tok#1:=#2\do#3{%
  \expandafter\def\expandafter\@fortmp\expandafter{#2}%
  \ifx\@fortmp\@empty \else
    \expandafter\@forloop@tok#2\@nil\@nil\@@#1{#3}%
  \fi}
\def\@forloop@tok#1#2#3\@@#4#5{%
  \def#4{#1}%
  \ifx #4\@nnil \else
    #5%
    \def#4{#2}%
    \ifx #4\@nnil \else
      #5\@iforloop@tok #3\@@#4{#5}%
  \fi\fi}
\def\@iforloop@tok#1#2\@@#3#4{%
  \def#3{#1}%
  \ifx #3\@nnil
    \expandafter\@fornoop
  \else
    #4\relax\expandafter\@iforloop@tok
  \fi
  #2\@@#3{#4}}
%
\newcommand*\My@extra@font{%
  \fontencoding{U}\fontfamily{MyriadPro-Extra}\selectfont}
\newcommand*\My@numerator@fig[1]{{\My@extra@font\My@@numerator@fig{#1}}}
\newcommand*\My@denominator@fig[1]{{\My@extra@font\My@@denominator@fig{#1}}}
\newcommand*\My@superior@fig[1]{{\My@extra@font\My@@superior@fig{#1}}}
\newcommand*\My@inferior@fig[1]{{\My@extra@font\My@@inferior@fig{#1}}}
\newcommand*\My@@numerator@fig[1]{%
  \@for@tok\@nf@fig:=#1\do{%
    \ifcase\@nf@fig
       \char'00%
    \or\char'01%
    \or\char'02%
    \or\char'03%
    \or\char'04%
    \or\char'05%
    \or\char'06%
    \or\char'07%
    \or\char'10%
    \or\char'11%
    \else
      \@latex@error{invalid argument to \string\My@@numerator@fig}%
    \fi
    }}
\newcommand*\My@@denominator@fig[1]{%
  \@for@tok\@nf@fig:=#1\do{%
    \ifcase\@nf@fig
       \char'20%
    \or\char'21%
    \or\char'22%
    \or\char'23%
    \or\char'24%
    \or\char'25%
    \or\char'26%
    \or\char'27%
    \or\char'30%
    \or\char'31%
    \else
      \@latex@error{invalid argument to \string\My@@denominator@fig}%
    \fi
    }}
\newcommand*\My@@superior@fig[1]{%
  \@for@tok\@nf@fig:=#1\do{%
    \ifcase\@nf@fig
       \char'60%
    \or\char'61%
    \or\char'62%
    \or\char'63%
    \or\char'64%
    \or\char'65%
    \or\char'66%
    \or\char'67%
    \or\char'70%
    \or\char'71%
    \else
      \@latex@error{invalid argument to \string\My@@superior@fig}%
    \fi
    }}
\newcommand*\My@@inferior@fig[1]{%
  \@for@tok\@nf@fig:=#1\do{%
    \ifcase\@nf@fig
       \char'100%
    \or\char'101%
    \or\char'102%
    \or\char'103%
    \or\char'104%
    \or\char'105%
    \or\char'106%
    \or\char'107%
    \or\char'110%
    \or\char'111%
    \else
      \@latex@error{invalid argument to \string\My@@inferior@fig}%
    \fi
    }}
%    \end{macrocode}
% |\Myensure@text| switches to text mode, if necessary.
%    \begin{macrocode}
\newcommand*\Myensure@text[1]{%
  \ifmmode
    \mdsy@text{#1}%
  \else
    #1%
  \fi}
%    \end{macrocode}
% |\smallfrac| and |\slantfrac| assemble numerical fractions.
%    \begin{macrocode}
\newcommand*\My@smallfrac[2]{%
  \leavevmode
  \setbox\@tempboxa
    \vbox{%
      \baselineskip\z@skip%
      \lineskip.25ex%
      \lineskiplimit-\maxdimen
      \ialign{\hfil##\hfil\crcr
              \vbox to 2.13ex{\vss\hbox{\My@numerator@fig{#1}}\vskip.68ex}\crcr
              \leavevmode\leaders\hrule height 1.1ex depth -1.01ex\hfill\crcr
              \vtop to 1ex{\vbox{}\hbox{\My@denominator@fig{#2}}\vss}\crcr
              \noalign{\vskip-1.47ex}}}%
  \dp\@tempboxa=0.49ex%
  \box\@tempboxa}
\newcommand*\My@slantfrac[2]{%
  {\My@extra@font\My@@numerator@fig{#1}\kern-0.05em/\kern-0.06em\My@@denominator@fig{#2}}}
\DeclareRobustCommand*\smallfrac[2]{\Myensure@text{\kern0.06em\My@smallfrac{#1}{#2}\kern0.09em}}
\DeclareRobustCommand*\slantfrac[2]{\Myensure@text{\kern0.06em\My@slantfrac{#1}{#2}\kern0.09em}}
%    \end{macrocode}
%
% \subsection{Additional symbols}
%
% Some symbols missing from MdSymbol can be taken from MyriadPro.
%
%    \begin{macrocode}
\if@My@Math@
  \let\hbar\undefined
  \DeclareMathSymbol{\hbar}             {\mathord}{letters}{'265}
  \DeclareMathSymbol{\uphbar}           {\mathord}{letters}{'255}
  \DeclareMathSymbol{\partial}          {\mathord}{letters}{'100}
  \DeclareMathSymbol{\uppartial}        {\mathord}{letters}{'300}
  \DeclareMathSymbol{\ell}              {\mathord}{letters}{'140}
  \DeclareMathSymbol{\upell}            {\mathord}{letters}{'340}
  \DeclareMathSymbol{\slashedzero}      {\mathord}{letters}{'257}
  \DeclareMathSymbol{\upimath}          {\mathord}{letters}{'373}
  \DeclareMathSymbol{\upjmath}          {\mathord}{letters}{'374}
  \DeclareMathSymbol{\varsmallint}      {\mathord}{letters}{'376}
  \DeclareMathSymbol{\openg}            {\mathalpha}{letters}{'267}
  \DeclareRobustCommand\lambdabar       {\middlebar\lambda}
  \DeclareRobustCommand\lambdaslash     {\middleslash\lambda}
\fi
%    \end{macrocode}
% Archaic Greek letters not provided by MyriadPro.
%    \begin{macrocode}
\if@My@Text@
  %\def\Qoppa{\reflectbox{P}}
  %\def\Sampi{\begingroup\fontfamily{cmr}\fontencoding{LGR}\selectfont\char23\endgroup}
  \let\Stigma\stigma

  % fix \r A
  \DeclareTextCompositeCommand{\r}{OT1}{A}
     {\leavevmode\setbox\z@\hbox{!}\dimen@\ht\z@\advance\dimen@-1ex%
     \ooalign{\hss\raise.67\dimen@\hbox{\char23}\hss\crcr A}}

  \DeclareEncodingSubset{TS1}{MyriadPro-LF}  {1}%
  \DeclareEncodingSubset{TS1}{MyriadPro-TLF} {1}%
  \DeclareEncodingSubset{TS1}{MyriadPro-OsF} {1}%
  \DeclareEncodingSubset{TS1}{MyriadPro-TOsF}{1}%
  \AtBeginDocument{
    \UndeclareTextCommand{\textvisiblespace}{T1}%
    \UndeclareTextCommand{\textcompwordmark}{T1}%
    \UndeclareTextCommand{\textsterling}{T1}%
    \UndeclareTextCommand{\j}{T1}%
    \UndeclareTextCommand{\j}{LY1}%
  }
\fi
%    \end{macrocode}
%
% \subsection{Integral symbols}
%
% We can also replace the integral signs from MdSymbol by those of MyriadPro.
% The following definitions provide this as an option.
%    \begin{macrocode}
\if@My@Math@
  \newcommand\My@Decl@Myriad@Ints{%
%    \end{macrocode}
% Replace |MdSymbolF| by |MySymbolFI|.
%    \begin{macrocode}
    \DeclareFontFamily{U}{MySymbolFI}{}
    \DeclareFontShape{U}{MySymbolFI}{m}{it}{
        <-6>  MySymbolFI\My@myriadint@opticals5
       <6-7>  MySymbolFI\My@myriadint@opticals6
       <7-8>  MySymbolFI\My@myriadint@opticals7
       <8-9>  MySymbolFI\My@myriadint@opticals8
       <9-10> MySymbolFI\My@myriadint@opticals9
      <10-12> MySymbolFI\My@myriadint@opticals10
      <12->   MySymbolFI\My@myriadint@opticals12
    }{}
    \DeclareFontShape{U}{MySymbolFI}{b}{it}{
        <-6>  MySymbolFI\My@myriadint@bold\My@myriadint@opticals5
       <6-7>  MySymbolFI\My@myriadint@bold\My@myriadint@opticals6
       <7-8>  MySymbolFI\My@myriadint@bold\My@myriadint@opticals7
       <8-9>  MySymbolFI\My@myriadint@bold\My@myriadint@opticals8
       <9-10> MySymbolFI\My@myriadint@bold\My@myriadint@opticals9
      <10-12> MySymbolFI\My@myriadint@bold\My@myriadint@opticals10
      <12->   MySymbolFI\My@myriadint@bold\My@myriadint@opticals12
    }{}
%    \end{macrocode}
%
%    \begin{macrocode}
    \DeclareSymbolFont{symbols}  {U}{MySymbolFI}{m}{it}
    \SetSymbolFont{symbols}{bold}{U}{MySymbolFI}{b}{it}
%    \end{macrocode}
% Make the original integral symbols available as |\var|\dots.
%    \begin{macrocode}
    \let\varint\tint
    \let\variint\tiint
    \let\variiint\tiiint
    \let\variiiint\tiiiint
    \let\varidotsint\tidotsint
    \let\varlandupint\tlandupint
    \let\varlanddownint\tlanddownint
    \let\varstrokedint\tstrokedint
    \let\varoint\toint
    \let\varoiint\toiint
    \let\varrcirclerightint\trcirclerightint
    \let\varlcirclerightint\tlcirclerightint
    \let\varrcircleleftint\trcircleleftint
    \let\varlcircleleftint\tlcircleleftint
    \let\varsumint\tsumint
%    \end{macrocode}
% Replace the symbols with the new integrals.
%    \begin{macrocode}
    \DeclareMathSymbol\tint            \mathop{symbols}{112}
    \DeclareMathSymbol\tiint           \mathop{symbols}{114}
    \DeclareMathSymbol\tiiint          \mathop{symbols}{116}
    \DeclareMathSymbol\tiiiint         \mathop{symbols}{118}
    \DeclareMathSymbol\tidotsint       \mathop{symbols}{120}
    \DeclareMathSymbol\tlandupint      \mathop{symbols}{122}
    \DeclareMathSymbol\tlanddownint    \mathop{symbols}{124}
    \DeclareMathSymbol\tstrokedint     \mathop{symbols}{126}
    \DeclareMathSymbol\toint           \mathop{symbols}{128}
    \DeclareMathSymbol\toiint          \mathop{symbols}{130}
    \DeclareMathSymbol\trcirclerightint\mathop{symbols}{132}
    \DeclareMathSymbol\tlcirclerightint\mathop{symbols}{134}
    \DeclareMathSymbol\trcircleleftint \mathop{symbols}{136}
    \DeclareMathSymbol\tlcircleleftint \mathop{symbols}{138}
    \DeclareMathSymbol\tsumint         \mathop{symbols}{140}
    \let\intop\tint
    \let\ointop\toint
  }
%    \end{macrocode}
%
%    \begin{macrocode}
  \My@load@integrals
\fi
%    \end{macrocode}
%
% \subsection{Open G support}
%
% We can replace the closed~$g$ with the open variant~$\openg$.
% The following definitions provide this as an option.
%    \begin{macrocode}
\if@My@Math@
  \My@Define@Open@g
\fi
%    \end{macrocode}
%
% \subsection{Logos}
%
% Correct logos.
%    \begin{macrocode}
\if@My@Text@
  \def\TeX{T\kern-.1667em\lower.4ex\hbox{E}\kern-.125emX\@}
  \DeclareRobustCommand{\LaTeX}{L\kern-.32em%
          {\sbox\z@ T%
           \vbox to\ht\z@{\hbox{\check@mathfonts
                                \fontsize\sf@size\z@
                                \math@fontsfalse\selectfont
                                A}%
                          \vss}%
          }%
          \kern-.15em%
          \TeX}
\fi
%    \end{macrocode}
%
% \subsection{AMS}
%
% Fix a bug in \pkg{amsmath.sty} which does not support
% math fonts without a skew char.
%    \begin{macrocode}
\def\macc@set@skewchar#1{%
  \begingroup
  \ifnum\mathgroup=\m@ne \let\@tempa\@ne
  \else
    \ifnum\skewchar\textfont\mathgroup=\m@ne \let\@tempa\@ne
    \else \let\@tempa\mathgroup
    \fi
  \fi
  \count@=\skewchar\textfont\@tempa
  \ifnum\count@=\m@ne
    \endgroup
    \def\macc@skewchar{}
  \else
    \advance\count@"7100
    \edef\@tempa{\endgroup
      \mathchardef\noexpand\macc@skewchar=\number\count@\relax}%
    \@tempa
  \fi
  #1%
}
%    \end{macrocode}
% Make the changes take effect. This concludes the main style file.
%    \begin{macrocode}
\if@My@Text@
  \normalfont
\fi
%</style>
%    \end{macrocode}
%
% \section{Support for character protrusion}
%
% The \pkg{microtype} configuration. All four MyriadPro families use the same
% file (cf.\ section~\ref{sec:fontdef}).
%    \begin{macrocode}
%<*mtcfg>
\SetProtrusion
  [ name     = MyriadPro-OT1-Roman ]
  { encoding = OT1,
    family   = {MyriadPro-OsF,MyriadPro-LF,MyriadPro-TOsF,MyriadPro-TLF},
    shape    = n }
  {
      A = {40,40},
      F = {  ,60},
      J = {90,  },
      K = {  ,50},
      L = {  ,60},
      T = {50,50},
      V = {40,40},
      W = {30,30},
      X = {50,50},
      Y = {50,50},
      k = {  ,60},
      r = {  ,80},
      t = {  ,100},
      v = {70,70},
      w = {40,40},
      x = {60,60},
      y = {70,70},
      ! = {70,180},
      ( = {60,30},    ) = {30,60},
      [ = {100,160},  ] = {160,100},
    {,} = {440,700},
      . = {660,700},
      : = {400,480},
      ; = {350,440},
      - = {700,700},
    \textendash       = {390,480},  \textemdash        = {220,270},
    \textquotedblleft = {380,250},  \textquotedblright = {250,380},
    \textquoteleft    = {670,450},  \textquoteright    = {450,670},
  }
%    \end{macrocode}
%
%    \begin{macrocode}
\SetProtrusion
  [ name     = MyriadPro-T1-Roman,
    load     = MyriadPro-OT1-Roman ]
  { encoding = T1,
    family   = {MyriadPro-OsF,MyriadPro-LF,MyriadPro-TOsF,MyriadPro-TLF},
    shape    = n }
  {
    023 = {  ,40}, % fft ligature
    032 = {  ,50}, % ft ligature
    191 = {30,30}, % Th ligature
    127 = {620,700}, % hyphen
    \AE = {40,  }, % AE
    \quotesinglbase = {670,670},  \quotedblbase   = {370,370},
    \guilsinglleft  = {500,360},  \guilsinglright = {360,500},
    \guillemotleft  = {320,230},  \guillemotright = {230,320},
  }
%    \end{macrocode}
%
%    \begin{macrocode}
\SetProtrusion
  [ name     = MyriadPro-OT1-Italic]
  { encoding = OT1,
    family   = {MyriadPro-OsF,MyriadPro-LF,MyriadPro-TOsF,MyriadPro-TLF},
    shape    = {it,sl} }
  {
      A = {120,50},
      B = {90,-50},
      C = {50,-60},
      D = {70,-30},
      E = {90,-50},
      F = {100,-40},
      G = {50,-60},
      H = {70,-40},
      I = {150,-90},
      J = {250,-130},
      K = {80,-50},
      L = {90,60},
      M = {60,-40},
      N = {70,-40},
      O = {70,-30},
      P = {70,-110},
      Q = {40,-40},
      R = {80,-50},
      S = {70,-70},
      T = {130,  },
      U = {70,-40},
      V = {120,30},
      W = {90,20},
      X = {50,  },
      Y = {160,  },
      Z = {50,-50},
      d = {60,-60},
      f = {  ,-190},
    027 = {  ,-70}, % ff ligature
      g = {-70,-70},
      i = {  ,-110},
    025 = {  ,-60}, % dotlessi
    028 = {  ,-60}, % fi ligature
    030 = {  ,-30}, % ffi ligature
      j = {-90,-150},
      p = {-40,  },
      r = {  ,80},
      t = {  ,100},
      v = {90,  },
      w = {60,10},
      x = {90,  },
      ! = {190,40},
      ( = {90,  },    ) = {90,  },
      [ = {90,90},    ] = {120,60},
    {,} = {210,680},
      . = {640,680},
      : = {380,430},
      ; = {   ,430},
      - = {750,750},
    \textquoteleft    = {690,140},  \textquoteright    = {470,230},
    \textendash       = {400,500},  \textemdash        = {220,280},
    \textquotedblleft = {520,130},  \textquotedblright = {520,130},
  }
%    \end{macrocode}
%
%    \begin{macrocode}
\SetProtrusion
  [ name     = MyriadPro-T1-Italic,
    load     = MyriadPro-OT1-Italic ]
  { encoding = T1,
    family   = {MyriadPro-OsF,MyriadPro-LF,MyriadPro-TOsF,MyriadPro-TLF},
    shape    = {it,sl} }
  {
    023 = {  ,40}, % fft ligature
    032 = {  ,50}, % ft ligature
    191 = {80,30}, % Th ligature
    127 = {660,750}, % hyphen
    \AE = {90,-40}, % AE
    131 = {80,-30}, % Dcaron
    132 = {70,-40}, % Ecaron
    156 = {80,-60}, % IJ
    \OE = {50,-30}, % OE
    188 = {  ,-80}, % ij
    184 = {70,70}, % ydieresis
    253 = {70,70}, % yacute
    \quotesinglbase = {220,700},  \quotedblbase   = {130,400},
    \guilsinglleft  = {500,180},  \guilsinglright = {350,350},
    \guillemotleft  = {310,110},  \guillemotright = {230,230},
  }
%    \end{macrocode}
%
%    \begin{macrocode}
\SetProtrusion
  [ name     = MyriadPro-other-Roman ]
  { encoding = {LGR,U,OT2,T2A,T2B,T2C,T5,X2},
    family   = {MyriadPro-OsF,MyriadPro-LF,MyriadPro-TOsF,MyriadPro-TLF},
    shape    = n }
  {
      ! = {70,180},
      ( = {60,30},    ) = {30,60},
      [ = {100,160},  ] = {160,100},
    {,} = {440,700},
      . = {660,700},
      : = {400,480},
      ; = {350,440},
      - = {700,700},
    \textendash       = {390,480},  \textemdash        = {220,270},
    \textquotedblleft = {380,250},  \textquotedblright = {250,380},
    \textquoteleft    = {670,450},  \textquoteright    = {450,670},
  }
\SetProtrusion
  [ name     = MyriadPro-other-Italic ]
  { encoding = {LGR,U,OT2,T2A,T2B,T2C,T5,X2},
    family   = {MyriadPro-OsF,MyriadPro-LF,MyriadPro-TOsF,MyriadPro-TLF},
    shape    = {it,sl} }
  {
      ! = {190,40},
      ( = {90,  },    ) = {90,  },
      [ = {90,90},    ] = {120,60},
    {,} = {210,680},
      . = {640,680},
      : = {380,430},
      ; = {   ,430},
      - = {750,750},
    \textquoteleft    = {690,140},  \textquoteright    = {470,230},
    \textendash       = {400,500},  \textemdash        = {220,280},
    \textquotedblleft = {520,130},  \textquotedblright = {520,130},
  }
%</mtcfg>
%    \end{macrocode}
%
% \section{Font definition files}
% \label{sec:fontdef}
%
% As all the font definitions look the same we introduce macros to ease the
% configuration. These macros are stored in the file
% \file{MyriadPro-FontDef.sty} which is included by every \acro{FD} file.
% Note that \file{MyriadPro-FontDef.sty} will be included several times and
% that we do not know in which context the code is executed. Therefore, we have
% to define all non-private commands as globals.
%
% \makeatletter
% \newcommand*\exampleconfig[2]{%
%   \begin{quote}\ttfamily
%   \expandafter\string\csname My@config@#1@#2\endcsname\par
%   \expandafter\expandafter\expandafter\strip@prefix
%   \expandafter\meaning\csname My@config@#1@#2\endcsname\par
%   \edef\@tempa{\My@UseConfig{#1}{#2}}%
%   \expandafter\strip@prefix\meaning\@tempa
%   \end{quote}%
% }%
% \makeatother
%
% Since this package should be loadable in an \acro{FD} file we have to
% avoid all |\preambleonly| commands.
% Therefore, we use |\ProvidesFile| instead of |\ProvidesPackage|.
%
% We add a guard so that this file is executed only once even if it is included
% multiple times.
%    \begin{macrocode}
%<*fontdef>
\ifx\My@DeclareFontShape\@undefined\else\endinput\fi
%    \end{macrocode}
% We distinguish between being loaded directly or via |\usepackage| in the
% preamble by checking |\@nodocument|.
%    \begin{macrocode}
\ifx\@nodocument\relax
  \input docstrip
\keepsilent
\askonceonly
\generate{
  \file{otfontdef.drv}{\from{otfontdef.dtx}{driver}}
  \file{otfontdef.sty}{\from{otfontdef.dtx}{otfd}}
}
\endbatchfile

\else
  \NeedsTeXFormat{LaTeX2e}
  \RequirePackage{otfontdef}
\fi
%    \end{macrocode}
% Reset |\escapechar| (which is set to $-1$ in \acro{FD} files) to make
% |\newcommand| work. The additional group does not harm; we have to
% make the important commands global anyway.
%    \begin{macrocode}
\ifx\@nodocument\relax
  \begingroup\escapechar`\\
\fi
%    \end{macrocode}
% These are the default values if it is impossible to process options.
%    \begin{macrocode}
\newcommand\My@option@opticals{noopticals}
\newcommand\My@option@fontset{smallfamily}
\newdimen\My@option@normalsize
\global\My@option@normalsize10pt
%    \end{macrocode}
% Whether we should adapt the configuration to the |\normalsize| of the
% document. This switch is only needed locally.
%    \begin{macrocode}
\newif\ifMy@option@normalsize
\My@option@normalsizetrue
%    \end{macrocode}
%
%    \begin{macrocode}
\ifx\@nodocument\relax\else
  \DeclareOption{noopticals} {\let\My@option@opticals\CurrentOption}
  \DeclareOption{smallfamily}{\let\My@option@fontset\CurrentOption}
  \DeclareOption{medfamily}  {\let\My@option@fontset\CurrentOption}
%  \DeclareOption{fullfamily} {\let\My@option@fontset\CurrentOption}
  \DeclareOption{normalsize}  {\My@option@normalsizetrue}
  \ExecuteOptions{smallfamily,noopticals,normalsize}
  \ProcessOptions\relax
\fi
%    \end{macrocode}
% The method to determine the main font size is inspired by \pkg{microtype}'s
% implementation.
%    \begin{macrocode}
\ifMy@option@normalsize
  \begingroup
  \def\set@fontsize#1#2#3#4\@nil{%
    \@defaultunits\global\My@option@normalsize#2pt\relax\@nnil}%
  \normalsize\@nil
  \endgroup
\fi
%    \end{macrocode}
% We use |\otf@makeglobal| from \pkg{otfontdef} to ``export'' the definitions
% that are needed globally.
%    \begin{macrocode}
\otf@makeglobal{My@option@opticals}
\otf@makeglobal{My@option@fontset}
\ifx\@nodocument\relax\else
  \PackageInfo{MyriadPro-FontDef}{%
    Configuration:\space\My@option@fontset,\space\My@option@opticals,\space
    normalsize=\the\My@option@normalsize}%
\fi
%    \end{macrocode}
%
% \subsubsection{Configuration database}
%
%    \begin{macrocode}
\newcount\My@config@cnt
\My@config@cnt=0
\newcommand\My@curr@config{My@config@\romannumeral\My@config@cnt}
%    \end{macrocode}
% These commands help in setting up the configuration database. They do not
% need to be global. But the config database itself has to be.
%
% |#3| is added to all instances listed in |#2| of configuration class |#1|.
% |#3| is read with \acro{NFSS} catcodes.
%    \begin{macrocode}
\newcommand\My@AddToConfig{%
  \begingroup
  \nfss@catcodes
  \expandafter\endgroup
  \My@AddToConfig@
}
\newcommand\My@AddToConfig@[3]{%
  \advance\My@config@cnt\@ne
  \@namedef{\My@curr@config}{#3}%
  \otf@makeglobal{\My@curr@config}
%<debug&show>\expandafter\show\csname\My@curr@config\endcsname
  \@for\My@tempa:=#2\do{%
    \@ifundefined{My@config@#1@\My@tempa}{%
      \@temptokena{}%
    }{%
      \@temptokena\expandafter\expandafter\expandafter
        {\csname My@config@#1@\My@tempa\endcsname}%
    }%
    \@expandtwoargs\@namedef{My@config@#1@\My@tempa}{%
      \the\@temptokena
      \expandafter\noexpand\csname\My@curr@config\endcsname
    }%
    \otf@makeglobal{My@config@#1@\My@tempa}% perhaps defer to only execute once
%<debug&show>\expandafter\show\csname My@config@#1@\My@tempa\endcsname
  }%
}
%    \end{macrocode}
%
% The following commands are used in the |Declare...Family| commands to access
% the previously built configuration database. They must be expandable. |#3| is
% used as a default if no entry is found in the database.
%    \begin{macrocode}
\newcommand*\My@UseConfig[2]{%
  \My@UseConfigOrDefault{#1}{#2}{}%
}
\newcommand*\My@UseConfigOrDefault[3]{%
  \@ifundefined{My@config@#1@#2}{#3}%
    {\@nameuse{My@config@#1@#2}}%
}
\newcommand*\My@TheConfig[2]{%
  \@ifundefined{My@config@#1@#2}{}{%
    \expandafter\noexpand\csname My@config@#1@#2\endcsname
  }%
}
\otf@makeglobal{My@UseConfig}
\otf@makeglobal{My@UseConfigOrDefault}
\otf@makeglobal{My@TheConfig}
%    \end{macrocode}
% The size range in the configuration has to be divided by the scaling
% factor to take the changed size into account because the scaling
% takes place after choosing the right combination. Provide
% calculation routine here.
%    \begin{macrocode}
\RequirePackage{fltpoint}
\fpDecimalSign{.}
\newcommand*{\My@calc@bsize}[2]{\fpDiv{#1}{#2}{\My@scaled}}
%    \end{macrocode}
% Here comes the configuration.
%    \begin{macrocode}
\My@calc@bsize{\My@s@capt}{8.5}
\My@calc@bsize{\My@s@text}{13.1}
\My@calc@bsize{\My@s@subh}{20}
\My@AddToConfig{opticals}{opticals}{
            <-\My@s@capt>  otf* [optical=Capt]
  <\My@s@capt-\My@s@text>  otf* [optical=Text]
  <\My@s@text-\My@s@subh>  otf* [optical=Subh]
  <\My@s@subh->            otf* [optical=Disp]
}
\My@AddToConfig{opticals}{noopticals}{
      <->     otf* [optical=Text]
}
\My@AddToConfig{opticals}{slides}{
      <->     otf* [optical=Capt]
}
%    \end{macrocode}
%
%    \begin{macrocode}
\My@calc@bsize{\My@s@semim}{6}
\My@AddToConfig{fontset/weight}{medfamily/m}{
             <-\My@s@semim> otf* [weight=Semibold]
  <\My@s@semim->            otf* [weight=Regular]
}
\My@AddToConfig{fontset/weight}{smallfamily/m}{
      <->     otf* [weight=Regular]
}
%
\My@calc@bsize{\My@s@bold}{6}
\My@AddToConfig{fontset/weight}{fullfamily/b,medfamily/b}{
            <-\My@s@bold>  otf* [weight=Bold]
  <\My@s@bold->            otf* [weight=Semibold]
}
\My@AddToConfig{fontset/weight}{smallfamily/b}{
      <->     otf* [weight=Bold]
}
%
\My@AddToConfig{weight}{eb}{
      <->     otf* [weight=Bold]
}
%    \end{macrocode}
%
%    \begin{macrocode}
\My@calc@bsize{\My@s@spac}{8}
\My@AddToConfig{shape}{n,it}{
      <-\My@s@spac>    otf* [spacing=l1]
}
\My@AddToConfig{encoding/shape}{U/n,U/it}{
      <->     otf* [spacing=]
}
%    \end{macrocode}
%
%    \begin{macrocode}
\My@AddToConfig{shape}{it}{
      <->     otf* MyriadPro-It
}
\My@AddToConfig{shape}{n}{
      <->     otf* MyriadPro
}
\My@AddToConfig{encoding/shape}{OML/it}{
      <->     otf* [figures=] MyriadPro-Mixed
}
\My@AddToConfig{encoding/shape}{OML/n}{
      <->     otf* [figures=] MyriadPro-French
}
\My@AddToConfig{scale}{scale}{
      <->     otf* [scale=\My@scaled]
}
%    \end{macrocode}
% Substitutions
%    \begin{macrocode}
\My@AddToConfig{sub:series} {sb}    {b}
\My@AddToConfig{sub:series} {bx}    {b}
\My@AddToConfig{sub:shape}  {sl}    {it}
%    \end{macrocode}
% Code for the last argument of |\DeclareFontShape|
%
% \subsubsection{Declaration of font families and shapes}
%
%    \begin{macrocode}
\newcommand*\My@DeclareFontShape[6][]{%
%    \end{macrocode}
% Check if any substitutions are specified.
%    \begin{macrocode}
  \edef\@tempa{%
    \My@UseConfig{sub:series}{#4}%
    \My@UseConfigOrDefault{sub:encoding/shape}{#2/#5}{%
      \My@UseConfig{sub:shape}{#5}}%
  }%
  \ifx\@tempa\@empty
%    \end{macrocode}
% Collect the configuration and declare the font shape. |\DeclareFontShape|
% fully expands its fifth argument (with our macros |\My@UseConfig| in it), but
% we have to retrieve the code for the sixth argument ourselves.
%    \begin{macrocode}
    \@temptokena={%
      \DeclareFontShape{#2}{#3-#6}{#4}{#5}{%
        \My@UseConfig{opticals}      {\My@option@opticals}%
        \My@UseConfig{fontset/weight}{\My@option@fontset/#4}%
        \My@UseConfig{weight}        {#4}%
        \My@UseConfig{encoding/shape}{#2/#5}%
        \My@UseConfig{shape}         {#5}%
        \My@UseConfig{scale}         {scale}%
      }}%
    \edef\@tempa{\the\@temptokena{\My@TheConfig{code:shape}{#5}}}%
    \@tempa
  \else
%    \end{macrocode}
% Generate the substitution. (All substitutions are silent at the moment.)
%    \begin{macrocode}
    \DeclareFontShape{#2}{#3-#6}{#4}{#5}{%
      <->ssub*#3-#6%
      /\My@UseConfigOrDefault{sub:series}{#4}{#4}%
      /\My@UseConfigOrDefault{sub:encoding/shape}{#2/#5}{%
        \My@UseConfigOrDefault{sub:shape}{#5}{#5}}%
    }{}%
  \fi
}
\otf@makeglobal{My@DeclareFontShape}
\otf@makeglobal{\string\My@DeclareFontShape}
%    \end{macrocode}
% |#2| contains the encoding, |#3| the family, and |#1| a list of figure
% versions (or |Extra|).
%    \begin{macrocode}
\newcommand*\My@DeclareLargeFontFamily[3][LF,OsF,TLF,TOsF]{%
  \My@DeclareFontFamily{#1}{#2}{#3}
    {m,sb,b,bx,eb} {n,it,sl}%
}
\newcommand*\My@DeclareSmallFontFamily[3][LF,OsF,TLF,TOsF]{%
  \My@DeclareFontFamily{#1}{#2}{#3}
    {m,sb,b,bx,eb} {n,it,sl}%
}
\newcommand*\My@DeclareMathFontFamily[3][TOsF]{%
  \My@DeclareFontFamily[\skewchar\font=255]{#1}{#2}{#3}
    {m,sb,b,bx,eb} {n,it}%
}
%    \end{macrocode}
% An additional macro |\csname\string\foo\endcsname| is generated by
% |\newcommand| for processing an optional argument of |\foo|.
%    \begin{macrocode}
\otf@makeglobal{My@DeclareLargeFontFamily}
\otf@makeglobal{\string\My@DeclareLargeFontFamily}
\otf@makeglobal{My@DeclareSmallFontFamily}
\otf@makeglobal{\string\My@DeclareSmallFontFamily}
\otf@makeglobal{My@DeclareMathFontFamily}
\otf@makeglobal{\string\My@DeclareMathFontFamily}
%    \end{macrocode}
%
%    \begin{macrocode}
\newcommand*\My@DeclareFontFamily[6][]{%
  \@for\My@variant:=#2\do{%
    \DeclareFontFamily {#3}{#4-\My@variant}{#1}%
  }%
  \My@DeclareFontShapes{#3}{#4}
    {#5} {#6} {#2}%
}
\otf@makeglobal{My@DeclareFontFamily}
\otf@makeglobal{\string\My@DeclareFontFamily}
%    \end{macrocode}
%
%    \begin{macrocode}
\newcommand*\My@DeclareFontShapes[5]{%
  \@for\My@series:=#3\do{%
    \@for\My@shape:=#4\do{%
      \@for\My@variant:=#5\do{%
        \My@DeclareFontShape{#1}{#2}{\My@series}{\My@shape}{\My@variant}%
      }%
    }%
  }%
}
\otf@makeglobal{My@DeclareFontShapes}
%    \end{macrocode}
% Adjust font dimension |#1| of the current font. The function in |#2| should
% replace the old value in dimen |\My@fontdimen| with a new one (which may
% depend on other parameters like |\f@size|).
%    \begin{macrocode}
\newdimen\My@fontdimen
\newcommand*\My@adjust@fontdimen[2]{%
  \My@fontdimen=\fontdimen#1\font
  #2%
  \fontdimen#1\font=\My@fontdimen
}
\otf@makeglobal{My@adjust@fontdimen}
%    \end{macrocode}
%
%    \begin{macrocode}
\ifx\@nodocument\relax
  \endgroup
\fi
%    \end{macrocode}
%    \begin{macrocode}
%<*debug>
\newcommand\old@DeclareFontFamily{}
\let\old@DeclareFontFamily\DeclareFontFamily
\renewcommand\DeclareFontFamily[3]{
  \begingroup\escapechar`\\%
  \edef\@tempa{\noexpand\DeclareFontFamily{#1}{#2}}%
  \@temptokena\expandafter{\@tempa{#3}}%
  \message{\the\@temptokena}%
  \endgroup
  \old@DeclareFontFamily{#1}{#2}{#3}%
}
\newcommand\old@DeclareFontShape{}
\let\old@DeclareFontShape\DeclareFontShape
\renewcommand\DeclareFontShape[6]{
  \begingroup\escapechar`\\%
  \edef\@tempa{\noexpand\DeclareFontShape{#1}{#2}{#3}{#4}{#5}}%
  \@temptokena\expandafter{\@tempa{#6}}%
  \message{\the\@temptokena}%
  \endgroup
  \old@DeclareFontShape{#1}{#2}{#3}{#4}{#5}{#6}%
}
%</debug>
%    \end{macrocode}
%
% We define font family aliases so that we can place all configurations for
% the MyriadPro family variants into one \pkg{microtype} file:
% \file{mt-MyriadPro.cfg}. We use \pkg{microtype}'s hook if \pkg{microtype}
% has not been loaded yet (which should be the case); otherwise we can
% execute the alias definitions directly.
%    \begin{macrocode}
\gdef\My@MicroType@Aliases{%
  \DeclareMicrotypeAlias{MyriadPro-LF}{MyriadPro}%
  \DeclareMicrotypeAlias{MyriadPro-OsF}{MyriadPro}%
  \DeclareMicrotypeAlias{MyriadPro-TLF}{MyriadPro}%
  \DeclareMicrotypeAlias{MyriadPro-TOsF}{MyriadPro}%
}
\@ifundefined{Microtype@Hook}{%
  \global\let\Microtype@Hook\My@MicroType@Aliases
}{%
  \g@addto@macro\Microtype@Hook{\My@MicroType@Aliases}%
}%
\@ifundefined{DeclareMicroTypeAlias}{}{\My@MicroType@Aliases}%
%</fontdef>
%    \end{macrocode}
%
% Using these macros the various \acro{FD} files become simple one-liners.
%
%    \begin{macrocode}
%<*fd>
\input{MyriadPro-FontDef.sty}%
%<Uextra>    \My@DeclareSmallFontFamily[Extra]{U}  {MyriadPro}
%<LGR>       \My@DeclareSmallFontFamily       {LGR}{MyriadPro}
%<LGI>       \My@DeclareSmallFontFamily       {LGI}{MyriadPro}
%<OT1>       \My@DeclareLargeFontFamily       {OT1}{MyriadPro}
%<T1>        \My@DeclareLargeFontFamily       {T1} {MyriadPro}
%<LY1>       \My@DeclareLargeFontFamily       {LY1}{MyriadPro}
%<T5>        \My@DeclareLargeFontFamily       {T5} {MyriadPro}
%<T2A>       \My@DeclareSmallFontFamily       {T2A}{MyriadPro}
%<T2B>       \My@DeclareSmallFontFamily       {T2B}{MyriadPro}
%<T2C>       \My@DeclareSmallFontFamily       {T2C}{MyriadPro}
%<TS1>       \My@DeclareLargeFontFamily       {TS1}{MyriadPro}
%<X2>        \My@DeclareSmallFontFamily       {X2} {MyriadPro}
%<OT2>       \My@DeclareSmallFontFamily       {OT2}{MyriadPro}
%<OML&tosf>  \My@DeclareMathFontFamily        {OML}{MyriadPro}
%<*OML&(lf|osf|tlf)>
  \@for\My@variant:=LF,TLF,OsF\do{%
    \DeclareFontFamily{OML}{MyriadPro-\My@variant}{\skewchar\font=255}
    \@for\My@series:=m,sb,b,bx,eb\do{%
      \@for\My@shape:=n,it\do{%
        \DeclareFontShape{OML}{MyriadPro-\My@variant}{\My@series}{\My@shape}%
          { <-> ssub*MyriadPro-TOsF/\My@series/\My@shape }{}
      }%
    }%
  }%
%</OML&(lf|osf|tlf)>
%</fd>
%    \end{macrocode}
%
% \CheckSum{0}
% \CharacterTable
%  {Upper-case    \A\B\C\D\E\F\G\H\I\J\K\L\M\N\O\P\Q\R\S\T\U\V\W\X\Y\Z
%   Lower-case    \a\b\c\d\e\f\g\h\i\j\k\l\m\n\o\p\q\r\s\t\u\v\w\x\y\z
%   Digits        \0\1\2\3\4\5\6\7\8\9
%   Exclamation   \!     Double quote  \"     Hash (number) \#
%   Dollar        \$     Percent       \%     Ampersand     \&
%   Acute accent  \'     Left paren    \(     Right paren   \)
%   Asterisk      \*     Plus          \+     Comma         \,
%   Minus         \-     Point         \.     Solidus       \/
%   Colon         \:     Semicolon     \;     Less than     \<
%   Equals        \=     Greater than  \>     Question mark \?
%   Commercial at \@     Left bracket  \[     Backslash     \\
%   Right bracket \]     Circumflex    \^     Underscore    \_
%   Grave accent  \`     Left brace    \{     Vertical bar  \|
%   Right brace   \}     Tilde         \~}
%
% \Finale
\endinput
% vim:set ft=tex:
